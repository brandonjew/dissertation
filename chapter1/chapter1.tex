\chapter{Introduction}
\section{Scope of Research}
Advances in sequencing technologies have paved the way for the generation of large-scale genomic datasets capturing genetic and transcriptomic variation across diverse sets of individuals \cite{Bycroft2018, GTEx_Consortium2020-xx}. These data allow researchers to elucidate the relationship between genetics, gene expression, and complex phenotypes \cite{Gusev2016}. Genome-wide association studies have identified a vast number of genetic loci associated with different traits \cite{gwas} and the identification of expression quantitative trait loci (eQTLs) bridges the gap between genotype and phenotype by elucidating the effects of genetic variation on gene expression \cite{Nica2013}. These associations serve as stepping stones for better understanding of human health and potential therapeutic targets \cite{Motsinger-Reif2013, walker2001pharmaceutical}. However, studies involving tissue-level gene expression are hindered by confounding factors that can both decrease power to detect associations and cause spurious results.

Many tissues in the human body consist of several cell types. For example, brain tissue often contains an assortment of neuronal and glial cell populations, each with distinct functions and gene expression profiles. RNA sequencing (RNA-seq) is often performed on whole tissues and analyses of these data can be confounded by cell-type heterogeneity across samples \cite{farahbod2020untangling}. Differences in cell type composition can be misinterpreted as differences in gene expression levels when measuring RNA across tissues. Furthermore, these mixtures can obscure cell-type-specific gene regulation that may be of interest \cite{Westra2015-vq, Shen-Orr2010-tg}. Single-cell approaches, such as single-cell RNA-seq (scRNA-seq) and single-nucleus RNA-seq (snRNA-seq), avoid these issues and provide insight into the gene expression of individual cells. However, these experiments remain costly, noisy, and difficult to scale compared to bulk RNA-seq \cite{lahnemann2020eleven}. Given the continuing utility of tissue-level RNA-seq datasets, it is important to accurately model variability in cell type proportions in these samples.

Regulation of gene expression also varies significantly across tissues \cite{GTEx_Consortium2020-xx}. Moreover, several tissues can have distinct roles in biological systems underlying a complex phenotype. For example, analyses of Type II diabetes have observed transcriptomic changes in blood \cite{Christodoulou2019}, adipose \cite{Miao2020}, and brain \cite{Zhou2019}. Therefore, it is imperative to measure gene expression across many tissues to fully understand gene regulation and its role in traits of interest. Analyzing gene expression across tissues, especially with growing sample sizes, requires approaches that can model this heterogeneity both accurately and efficiently.

Finally, the relationship between gene expression and genetic variation is highly complex \cite{robert2018exploring} and remains to be fully understood. Many eQTLs have been identified and are thought to influence gene expression through interactions with regulatory regions of the genome, such as enhancers and promoters \cite{Garieri2017}. Alternative mechanisms of genetic influences on gene regulation have been hypothesized, specifically in the context of selective pressures causing non-random X chromosome inactivation \cite{Migeon1998-gc}. To better understand human biology, it is important to validate these hypotheses and identify the extent of their influence on regulation of gene expression. 

\section{Contributions and Overview}
In this dissertation, we introduce computational and statistical methods to accurately model tissue-level expression data with heterogeneity in cell composition and tissue identity. Furthermore, we present analyses of expression data that yields further insight into non-random X chromosome inactivation as an additional mechanism for gene regulation with several interesting implications in the context of X-linked phenotypes.

There is large interest in estimating cell type proportions from tissue-level RNA-seq data. These estimates allow researchers to account for this potential confounding factor in analyses of gene expression data, such as eQTL and differential expression studies. Furthermore, cell type proportions can be used with additional methods to identify cell-type-specific associations from tissue-level data \cite{Shen-Orr2010-tg}. In Chapter 2, we describe Bisque, an approach for accurately estimating cell type proportions from bulk RNA-seq data. This method utilizes available single-cell data as a reference for cell-type-specific expression while accounting for technological biases between bulk and single-cell RNA-seq.

In Chapter 3, we present an efficient linear mixed model (LMM) for the analysis of massive datasets measuring multiple contexts across a set of individuals. This method, mcLMM, models all contexts jointly in a meta-analytic framework to improve power to detect associations. It can be applied to large expression datasets with measurements across several tissues, such as the GTEx dataset \cite{GTEx_Consortium2020-xx}, to efficiently identify genetic variants that influence gene regulation in specific tissues or across several tissues. We further demonstrate the utility of this method by performing a multi-trait genome-wide association study across hundreds of thousands of individuals in the UK Biobank \cite{Bycroft2018} using minimal computational resources.

In Chapter 4, we analyze the GTEx dataset \cite{GTEx_Consortium2020-xx} to support the hypothesis of selective pressures influencing non-random X chromosome inactivation \cite{Migeon1998-gc} and quantify the extent of this effect in the female population. We estimate skewing in X inactivation from bulk RNA-seq data measured across non-diseased tissue samples and identify several genetic factors that are significantly associated with preferential inactivation of a haplotype, such as variation associated with increased deleteriousness and decreased proliferation. Furthermore, we identify common genetic variants in specific loci that contribute to skewed X chromosome inactivation. We highlight the implications of non-random skew in X chromosome inactivation, such as decreased penetrance or dampened effects of genetic variation on preferentially inactivated haplotypes.